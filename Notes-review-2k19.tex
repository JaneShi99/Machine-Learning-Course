

\documentclass[12pt]{article} 


%%%%%%% a few packages
\usepackage{fullpage}
\usepackage{todonotes}
\usepackage{color}
\usepackage{hyperref} % for the URL
\usepackage{pst-tree} % for the trees
\usepackage{verbatim} 
\usepackage{ifthen} 
\usepackage{amsmath}
\usepackage{listings}  
\usepackage{amssymb}
\usepackage{array}
\usepackage{multicol}
\definecolor{dkgreen}{rgb}{0,0.6,0}
\definecolor{gray}{rgb}{0.5,0.5,0.5}
\definecolor{mauve}{rgb}{0.58,0,0.82}
\lstset{frame=tb,
	language=Python,
	aboveskip=3mm,
	belowskip=3mm,
	showstringspaces=false,
	columns=flexible,
	basicstyle={\small\ttfamily},
	numbers=none,
	numberstyle=\tiny\color{gray},
	keywordstyle=\color{blue},
	commentstyle=\color{dkgreen},
	stringstyle=\color{mauve},
	breaklines=true,
	breakatwhitespace=true,
	tabsize=3
}


%%%%%% for pst-tree, define a node that ALWAYS has the same width (here "99")
\newlength{\nodeLength}
\newcommand{\Node}{A}
\newcommand{\setnode}[1]{ \settowidth{\nodeLength}{#1}
  \renewcommand{\Node}[1]{ \Tcircle{\makebox[\nodeLength]{##1}} }
}
\setnode{99}

%%%%%%%%%%%%%%%%%%%%% algo.sty copied over %%%%%%%%%%%%%%%%%%%%%
\newcounter{algorithmeligne}
\newcommand{\instr}[1]{%\underline
                        {\bf #1}}
\newcommand{\nomproc}[1]{{\rm\bf #1}}
\newenvironment{algorithme} { \setcounter{algorithmeligne}{0}
        \newcommand{\lign}{\stepcounter{algorithmeligne}
                \>{\footnotesize \arabic{algorithmeligne}.}\> }
        \begin{center}
        \begin{tabular}{|c|}
        \hline
        \begin{minipage}{1cm} \small \, \ \ \\[-11mm] \begin{tabbing}
\=\hskip1cm\=\qquad\=\qquad\=\qquad\=\qquad\=\qquad\=\qquad\=\qquad\=\qquad\=\qquad\=\qquad\=\qquad\\\kill 
        }
        {
        \end{tabbing}
        \ \ \\[-8mm]
        \end{minipage}
        \\\hline
        \end{tabular}
        \end{center}
        }


\setlength{\parskip}{0.25cm plus 4mm minus 3mm}

\renewcommand\labelitemi{-}

\begin{document}

\begin{center}
{\Large \bf Coursera- Machine Learning}\\
\vspace{3mm}
{\Large \bf May 2019}\\
\vspace{3mm}
{\Large \bf Taught by Prof. Andrew Ng}\\
\vspace{3mm}
\textbf{Janeshi99}\\
\end{center}


\section*{Summary}
Supervised learning
\begin{itemize}
	\item linear regression, logistic regression, neural network, SVMs
\end{itemize}
Unsupervised learning
\begin{itemize}
	\item k-means, PCA, Anomaly detection
\end{itemize}
Special applications/special topics
\begin{itemize}
	\item Recommender systems, large scale machine learning
	
\end{itemize}
Advice for building a machine learning system
\begin{itemize}
	\item bias/variance, regularization, deciding what to work next, evaluation of a learning algorithm, learning curves, error analysis, ceiling analysis
\end{itemize}


\section*{Week 1}
\underline{Intro}\\
Definition of ML\\
\begin{itemize}
	\item A program learns from experience (E) w.r.t task(T) and performance measure (P) if its performance on T improves with more E.\\
	\item With supervised learning, we know what our answers are as a relation of input and output. But with unsupervised learning, we have little idea about the result.\\
\end{itemize}

\underline{Cost function}
\begin{itemize}
	\item \[h_{\theta}(x) = \theta_0 + \theta_1x\]
	 our goal is to minimize the cost function, which is calculated as square error
	 \[\min_{\theta_0,\theta_1} J(\theta_0,\theta_1)\]
	\item where the error function is defined as
	 \[J(\theta_0,\theta_1) = \frac{1}{2m} \sum_{}^{}(h_\theta(x^{(i)} - y ^{(i)})^2)\]
\end{itemize}

\underline{Linear regression}
\begin{center}
	\begin{itemize}
		\item 	Repeat until converge\{
		$  \theta_j := \theta_j - \alpha\frac{\partial}{\partial \theta_j}J(\theta_0,\theta_1) \text{ for } j = 0,1 $ \}	
		\item  Note that the update is \underline{simultaneous} :
		\item 
		\[ \text{temp}_0 := \theta_0 - \alpha\frac{\partial}{\partial \theta_0}J(\theta_0,\theta_1) \] 
		\[ \text{temp}_1 := \theta_1 - \alpha\frac{\partial}{\partial \theta_1}J(\theta_0,\theta_1) \]
		\[\theta_0 := \text{temp}_0 \]
		\[\theta_1 := \text{temp}_1\]
		\item if we compute the derivative we get
		Repeat until converge\{
		
		\[ \theta_0 := \theta_0 - \alpha\frac{1}{m}\sum_{i=1}^{m}(h_{\theta}(x_i)-y_i) \] 
		\[ \theta_1 := \theta_1 - \alpha\frac{1}{m}\sum_{i=1}^{m}((h_{\theta}(x_i)-y_i) x_i)\] 
		
		\}	
		
		\item $\alpha$ is the learning rate. 
		\item we use linear regression algorithm to updates the parameters until we arrive at the minimal cost.
	\end{itemize}

\end{center}

\section*{Week 2}
\underline{Multi-feature linear regression}\\
\begin{itemize}
\item Hypothesis
\[h_\theta (x) = \theta_0+ \theta_1x_1 +\ldots \theta_nx_n\]
\item convenience $\forall x$, $x_0=1$, so that $h_\theta= \sum_{i=0}^{n}\theta_ix_i$
\item 
$$ x= \begin{bmatrix}
	x_0 \\
	x_1 \\
	\vdots \\
	x_n
\end{bmatrix}
 \in \mathbb{R} ^{n+1}
\text{ and that } \theta = \begin{bmatrix}
	\theta_0 \\
	\theta_1 \\
	\vdots \\
	\theta_n
\end{bmatrix}
$$
\item Hypothesis can be represented as
\[h_\theta(x)=\theta^T x \text{ or } <\theta,x>\]
\item The parameter we're estimating here is $\theta$
\item Cost function 
\[J(\theta) = \frac{1}{2m}\sum_{i=1}^{m}((h_\theta)x^{(i)})-y^{(i)})^2\]
\item Gradient descent
\[repeat \{ \theta_j := \theta_j - \alpha\frac{1}{m}\sum_{i=1}^{m} (h_\theta(x^{(i)})-y^{(i)})x_j^{(i)}  \text{ ,simultaneously update }\theta_0\ldots \theta_j\}\]
\item When working with gradient descent in practice, we should... consider
\item Feature scaling:\\
Make sure features are in a similar scale, so that each values are roughly between $[-3,3]$
\item Mean normalization:
Replace all $x_i$ (except for $x_0$ )with $x_i-\mu_i$ so that the mean is roughly $0$.\\
\[x_i\leftarrow \frac{x_i-u_i}{s_i}\]
\item Note that $J$ should always decresase w.r.t to the number of iteration. If it ever increases, that means our $\alpha$, the step param, is too large. We would want to decrease $\alpha$.\\
\item Pick $\epsilon$ for the convergence threshold value.
\item Tip: in order to choose $\alpha$, try a range of values. Example: choosing based on a logarithmic scale: \[0.001,0.003,0.01,0.03,\ldots\]

\end{itemize}
\underline{feature \& polynomial regression}
\begin{itemize}
	\item We can combine multiple features into one and change the behaviour of the hypothesis.
	\item For example we can combine $x_1. x_2$ into a polynomial term by defining that $x_3=x_1*x_2$.
	\item polynomial regression, instead of linear, we make it quadratic or cubic to tune the hypothesis
	\[h_\theta(x) = \theta_0 + \theta_1x_1 + \theta_2x_2^2 + \theta_3\sqrt{x}\]
	Keep in mind that feature scaling is still very important.
	
\end{itemize}

\underline{Normal equation: computing parameters analytically}
\begin{itemize}
\item Define $X$ as the design matrix. That is 
$$ \text{if } x^{(i)}= \begin{bmatrix}
x_0^{(i)} \\
x_1^{(i)} \\
\vdots \\
x_n^{(i)} \end{bmatrix}
\text{ then we have }
X =
\left[
\begin{array}{ccc}
-& (x^{(1)})^T & - \\
- & (x^{(2)})^T & - \\
& \vdots    &          \\
- & (x^{(n)})^T & -
\end{array}
\right]
$$
\item Optimum $\theta$ given by $\theta = (X^TX)^{-1}X^Ty$
\item With normal equation, you don't need feature scaling. 
	\begin{multicols}{2}
	\textit{Gradient descent}:\\
	$\alpha$ needs to be chosen\\
	needs many iterations\\
	works well even when $n$ is large\\
	
		\columnbreak
		
		\textit{Normal equation}:\\
		no need to choose $\alpha$\\
		no iterations needed\\
		computing $(X^TX)^{-1}$ takes $O(n^3)$\\
		performs slow with large $n (n\geq 10,000)$
	\end{multicols}
\item Note: what if $X^TX$ is non-invertible? Then we use the $pinv$ to generate the pseudo-inverse.
\end{itemize} 
\underline{vectorization} helps to compute vectors faster.

\section{Week 3}
\underline{Binary classification problems}
\begin{itemize}
	\item each element $y$ belongs to negative class ($0$) or positive class ($1$).
\end{itemize}
\underline{Logistic regression}
	\begin{itemize}
		\item We want $0\leq h_\theta(x) \leq 1$\\
		$h_\theta(x) = g(\theta^Tx)$ where $g$ is the sigmoid function\\
		$g(z) = \frac{1}{1+e^{-z}}$
	\end{itemize}
\underline{decision boundary}
\begin{itemize}

\item The decision boundary is the line that separates area where $y=0$ or $y=1$.
\item Note that 
\[h_\theta(x)\geq0.5\iff \theta^TX\geq 0 \to y=1\]
\[h_\theta(x)<0.5\iff \theta^TX< 0 \to y=0\]
	\item we can also work with non-linear decision boundaries
\end{itemize}
\underline{Logistic regression model}
\begin{itemize}
	
	\item The training set will be $\{((x^{(1)},y^{(1)}), \ldots (x^{(m)},y^{(m)})))\}$\\
	There are $m$ examples, and for $\forall x, x\in \mathbb{R}^{n+1}$, $x_0=1$, $y\in\{0,1\}$\\
	$h_\theta(x) = \frac{1}{1 + e^{-\theta^Tx}}$
	\item We realize that lin.reg. will not give you a convex function but we \underline{want} a convex function. This brings us to construct a good cost function.
	\item 
	 \[
	cost(h_\theta(x),y)=\left\{
	\begin{array}{ll}
	-\log(h_\theta(x)) &\text{ if }y=1\\
	-\log(1-h_\theta(x))& \text{ if }y=0
	\end{array}
	\right.
	\]
	\item For example if $y=1$ then if $x=1$ we have cost=$0$. And as hypothesis approach $0$, cost approaches $\infty$ so we're penalized. Similar with the other situation.
	\item This gives us a convex and local optimum free function.
	\item The \underline{uncompressed cost function } is:
	\[\text{cost } (h_\theta(x),y) = -y\log(h_\theta(x))-(1-y)log(1-h_\theta(x))\]
	\item The total cost unction $J$:
	\[j(\theta) = \frac{1}{m}\sum_{i=1}^{m} \text{cost } (h_\theta(x),y) \]
	\item The gradient descent algorithm is essetially the same but referring to a different hypothsis which is $h_\theta(x)$ that now refers to the sigmoid function\\
	\end{itemize}
\underline{The vectorized implementation}

\begin{itemize}
	\item $h=g(X\theta)$, which computes the quantity $h_\theta(x^{(i)})$
	\item
	$J(\theta) = \frac{1}{m}(-y^T\log(h)-(1-y)^T\log(1-h))$
\end{itemize}

\underline{Gradient descent}
\begin{itemize}
	\item Idea is to re-arrange the vectors until it's easier to type into matlab.
	\item Reminder that the matrix $X$ looks like this: 
	$$X =
	\left[
	\begin{array}{ccc}
	-& (x^{(1)})^T & - \\
	- & (x^{(2)})^T & - \\
	& \vdots    &          \\
	- & (x^{(m)})^T & -
	\end{array}
	\right]
	$$
	\item X is a $m\times n$ matrix (ignoring the extra leftmost column of $1$s). $\theta$ is a $n\times1$ vector, which makes $X^T\theta$ a $m\times1$ vector which yields the answer.
	\item Left off at the spot where we have two theta equations.
\end{itemize}
\end{document}