

\documentclass[12pt]{article} 


%%%%%%% a few packages
\usepackage{fullpage}
\usepackage{todonotes}
\usepackage{color}
\usepackage{hyperref} % for the URL
\usepackage{pst-tree} % for the trees
\usepackage{verbatim} 
\usepackage{ifthen} 
\usepackage{amsmath}
\usepackage{listings}  


\definecolor{dkgreen}{rgb}{0,0.6,0}
\definecolor{gray}{rgb}{0.5,0.5,0.5}
\definecolor{mauve}{rgb}{0.58,0,0.82}
\lstset{frame=tb,
	language=Python,
	aboveskip=3mm,
	belowskip=3mm,
	showstringspaces=false,
	columns=flexible,
	basicstyle={\small\ttfamily},
	numbers=none,
	numberstyle=\tiny\color{gray},
	keywordstyle=\color{blue},
	commentstyle=\color{dkgreen},
	stringstyle=\color{mauve},
	breaklines=true,
	breakatwhitespace=true,
	tabsize=3
}


%%%%%% for pst-tree, define a node that ALWAYS has the same width (here "99")
\newlength{\nodeLength}
\newcommand{\Node}{A}
\newcommand{\setnode}[1]{ \settowidth{\nodeLength}{#1}
  \renewcommand{\Node}[1]{ \Tcircle{\makebox[\nodeLength]{##1}} }
}
\setnode{99}

%%%%%%%%%%%%%%%%%%%%% algo.sty copied over %%%%%%%%%%%%%%%%%%%%%
\newcounter{algorithmeligne}
\newcommand{\instr}[1]{%\underline
                        {\bf #1}}
\newcommand{\nomproc}[1]{{\rm\bf #1}}
\newenvironment{algorithme} { \setcounter{algorithmeligne}{0}
        \newcommand{\lign}{\stepcounter{algorithmeligne}
                \>{\footnotesize \arabic{algorithmeligne}.}\> }
        \begin{center}
        \begin{tabular}{|c|}
        \hline
        \begin{minipage}{1cm} \small \, \ \ \\[-11mm] \begin{tabbing}
\=\hskip1cm\=\qquad\=\qquad\=\qquad\=\qquad\=\qquad\=\qquad\=\qquad\=\qquad\=\qquad\=\qquad\=\qquad\\\kill 
        }
        {
        \end{tabbing}
        \ \ \\[-8mm]
        \end{minipage}
        \\\hline
        \end{tabular}
        \end{center}
        }


\setlength{\parskip}{0.25cm plus 4mm minus 3mm}



\begin{document}

\begin{center}
{\Large \bf Coursera- Machine Learning}\\
\vspace{3mm}
{\Large \bf May 2019}\\
\vspace{3mm}
{\Large \bf Taught by Prof. Andrew Ng}\\
\vspace{3mm}
\textbf{Janeshi99}\\
\end{center}


\section*{Summary}
Supervised learning
\begin{itemize}
	\item linear regression, logistic regression, neural network, SVMs
\end{itemize}
Unsupervised learning
\begin{itemize}
	\item k-means, PCA, Anomaly detection
\end{itemize}
Special applications/special topics
\begin{itemize}
	\item Recommender systems, large scale machine learning
	
\end{itemize}
Advice for building a machine learning system
\begin{itemize}
	\item bias/variance, regularization, deciding what to work next, evaluation of a learning algorithm, learning curves, error analysis, ceiling analysis
\end{itemize}


\section*{Week 1}
\underline{Intro}\\
Definition of ML\\
\begin{itemize}
	\item A program learns from experience (E) w.r.t task(T) and performance measure (P) if its performance on T improves with more E.\\
	\item With supervised learning, we know what our answers are as a relation of input and output. But with unsupervised learning, we have little idea about the result.\\
\end{itemize}

\underline{Cost function}
\begin{itemize}
	\item \[h_{\theta}(x) = \theta_0 + \theta_1x\]
	 our goal is to minimize the cost function, which is calculated as square error
	 \[\min_{\theta_0,\theta_1} J(\theta_0,\theta_1)\]
	\item where the error function is defined as
	 \[J(\theta_0,\theta_1) = \frac{1}{2m} \sum_{}^{}(h_\theta(x^{(i)} - y ^{(i)})^2)\]
\end{itemize}

\underline{Linear regression}
\begin{center}
	repeat until converge\{
		\[\]\}	
\end{center}

\end{document}